\documentclass[12pt,a4paper,oneside, titlepage]{report}

% --- Polices et Langues (Correction pour LuaLaTeX) ---
\usepackage[french]{babel} % Utilisation de 'french' au lieu de 'frenchb'
\usepackage{fontspec}      % Requis pour LuaLaTeX
\setmainfont{Latin Modern Roman} % Police standard propre

\usepackage{hyperref} 
\usepackage{color}
\usepackage{graphicx}
\usepackage{fancyhdr}
\usepackage{amsthm} % Ajouté pour gérer l'environnement demonstration/proof

\pagestyle{fancy}

% --- Théorèmes et environnements ---
\newtheorem{defi}{Définition}[section]
\newtheorem{note}{Note}[section]
\newtheorem{proprietet}{Propriété}[section]
\newtheorem{exemple}{Exemple}[section]
\newtheorem{corollaire}{Corollaire}[section]
\newtheorem{rem}{Remarque}[section]
\newtheorem{thm}{Théorème}[section]
\newtheorem{illustration}{Illustration}[section]

% Correction de l'environnement démonstration
\newenvironment{demonstration}{\begin{proof}[\textnormal{\textbf{Preuve.}}]}{\end{proof}}

\definecolor{gris}{gray}{0.45}
\setlength{\parindent}{1cm}

% --- En-têtes et pieds de page ---
\renewcommand{\chaptermark}[1]{\markright{\thechapter\ #1}}
\fancyhf{} 
\fancyhead[R]{\thepage}
\fancyhead[L]{\textsl{\leftmark}} 
\renewcommand{\headrulewidth}{0pt}
\renewcommand{\footrulewidth}{0pt} 

\fancypagestyle{plain}{
  \fancyhead{} 
  \fancyhead[R]{\thepage}
  \renewcommand{\headrulewidth}{0pt}
}

\begin{document}

\pagenumbering{roman}
\chapter*{Remerciements}
\renewcommand{\leftmark}{REMERCIEMENTS}

Nous remercions ...\\

\newpage
\renewcommand{\leftmark}{TABLE DES MATIÈRES}
\thispagestyle{fancy}
\tableofcontents

\newpage
\pagenumbering{arabic}
\renewcommand{\leftmark}{INTRODUCTION}
\chapter{Introduction}

\section{Contexte et motivation}
Contenu à rédiger.

\section{Problématique}
Contenu à rédiger.

\section{Objectifs du mémoire}
Contenu à rédiger.

\section{Questions de recherche}
Contenu à rédiger.

\section{Méthodologie générale}
Contenu à rédiger.

\section{Organisation du mémoire}
Contenu à rédiger.

\newpage

\chapter{Fondements et évolution du CI/CD}
\renewcommand{\leftmark}{CHAPITRE \thechapter.~~Fondements et évolution du CI/CD}

\section{Workflow de développement logiciel}
Contenu à rédiger.

\section{Origines et premières implémentations du CI}
Contenu à rédiger.

\section{Évolution vers l’intégration, la livraison et le déploiement continus}
Contenu à rédiger.

\section{Intégration du CI/CD dans les pratiques DevOps}
Contenu à rédiger.

\section{Automatisation des workflows et Configuration as Code}
Contenu à rédiger.

\newpage

\chapter{État de l’art : revue de la littérature scientifique}
\renewcommand{\leftmark}{CHAPITRE \thechapter.~~État de l’art}

\section{Méthodologie de revue de littérature}
Contenu à rédiger.

\section{Bénéfices du CI/CD observés dans la littérature}
Contenu à rédiger.

\section{Défis et limitations du CI/CD}
Contenu à rédiger.

\section{Pratiques recommandées et modèles existants}
Contenu à rédiger.

\section{Synthèse critique de la littérature}
Contenu à rédiger.

\newpage

\chapter{État de la pratique : outils et plateformes CI/CD}
\renewcommand{\leftmark}{CHAPITRE \thechapter.~~État de la pratique}

\section{Outils CI/CD historiques et standalone}
Contenu à rédiger.

\section{Plateformes de développement collaboratif intégrant le CI/CD}
\subsection{GitHub}
Contenu à rédiger.

\subsection{GitLab}
Contenu à rédiger.

\subsection{Bitbucket}
Contenu à rédiger.

\subsection{Gitea}
Contenu à rédiger.

\section{Modèles de pipelines et langages de configuration}
Contenu à rédiger.

\section{Fonctionnalités communes et divergences entre plateformes}
Contenu à rédiger.

\section{Synthèse comparative de l’état de la pratique}
Contenu à rédiger.

\newpage

\chapter{Méthodologie de comparaison}
\renewcommand{\leftmark}{CHAPITRE \thechapter.~~Méthodologie de comparaison}

\section{Objectifs de la comparaison}
Contenu à rédiger.

\section{Critères d’analyse des systèmes CI/CD}
Contenu à rédiger.

\section{Méthode d’évaluation qualitative}
Contenu à rédiger.

\section{Métriques quantitatives utilisées}
Contenu à rédiger.

\section{Menaces à la validité}
Contenu à rédiger.

\newpage

\chapter{Étude de cas : implémentation expérimentale}
\renewcommand{\leftmark}{CHAPITRE \thechapter.~~Étude de cas}

\section{Description du projet support}
Contenu à rédiger.

\section{Choix des plateformes CI/CD étudiées}
Contenu à rédiger.

\section{Implémentation des pipelines CI/CD}
Contenu à rédiger.

\section{Résultats expérimentaux}
Contenu à rédiger.

\section{Analyse comparative basée sur l’étude de cas}
Contenu à rédiger.

\newpage

\chapter{Discussion}
\renewcommand{\leftmark}{CHAPITRE \thechapter.~~Discussion}

\section{Interprétation des résultats}
Contenu à rédiger.

\section{Comparaison avec la littérature existante}
Contenu à rédiger.

\section{Implications pour les praticiens}
Contenu à rédiger.

\section{Limites du travail}
Contenu à rédiger.

\newpage

\chapter{Conclusion et perspectives}
\renewcommand{\leftmark}{CHAPITRE \thechapter.~~Conclusion et perspectives}

\section{Synthèse des contributions}
Contenu à rédiger.

\section{Réponses aux questions de recherche}
Contenu à rédiger.

\section{Perspectives de recherche future}
Contenu à rédiger.

% Bibliographie
\bibliographystyle{plain} % 'latex8' est souvent un fichier .bst externe, 'plain' est plus sûr pour tester
\bibliography{biblio}

\newpage
\appendix
\addcontentsline{toc}{chapter}{Annexes}

\chapter{Première annexe}
\renewcommand{\leftmark}{ANNEXE \thechapter.~~Première annexe}
\label{annexe1}

\chapter{Deuxième annexe}
\renewcommand{\leftmark}{ANNEXE \thechapter.~~Deuxième annexe}
\label{annexe2}

\end{document}