\documentclass[12pt,a4paper,oneside, titlepage]{report}

% --- Polices et Langues (Correction pour LuaLaTeX) ---
\usepackage[french]{babel} % Utilisation de 'french' au lieu de 'frenchb'
\usepackage{fontspec}      % Requis pour LuaLaTeX
\setmainfont{Latin Modern Roman} % Police standard propre

\usepackage{hyperref} 
\usepackage{color}
\usepackage{graphicx}
\usepackage{fancyhdr}
\usepackage{amsthm} % Ajouté pour gérer l'environnement demonstration/proof

\pagestyle{fancy}

% --- Théorèmes et environnements ---
\newtheorem{defi}{Définition}[section]
\newtheorem{note}{Note}[section]
\newtheorem{proprietet}{Propriété}[section]
\newtheorem{exemple}{Exemple}[section]
\newtheorem{corollaire}{Corollaire}[section]
\newtheorem{rem}{Remarque}[section]
\newtheorem{thm}{Théorème}[section]
\newtheorem{illustration}{Illustration}[section]

% Correction de l'environnement démonstration
\newenvironment{demonstration}{\begin{proof}[\textnormal{\textbf{Preuve.}}]}{\end{proof}}

\definecolor{gris}{gray}{0.45}
\setlength{\parindent}{1cm}

% --- En-têtes et pieds de page ---
\renewcommand{\chaptermark}[1]{\markright{\thechapter\ #1}}
\fancyhf{} 
\fancyhead[R]{\thepage}
\fancyhead[L]{\textsl{\leftmark}} 
\renewcommand{\headrulewidth}{0pt}
\renewcommand{\footrulewidth}{0pt} 

\fancypagestyle{plain}{
  \fancyhead{} 
  \fancyhead[R]{\thepage}
  \renewcommand{\headrulewidth}{0pt}
}

\begin{document}

\pagenumbering{roman}
\chapter*{Remerciements}
\renewcommand{\leftmark}{REMERCIEMENTS}

Nous remercions ...\\

\newpage
\renewcommand{\leftmark}{TABLE DES MATIÈRES}
\thispagestyle{fancy}
\tableofcontents

\newpage
\pagenumbering{arabic}
\renewcommand{\leftmark}{INTRODUCTION}
\chapter{Introduction}
Mettez l'introduction ici. Expliquez le contexte du travail, et les objectifs du travail.

% Note: Assurez-vous d'avoir un fichier biblio.bib si vous utilisez \cite
% Avant de commencer la rédaction d'un projet ou un mémoire, lisez d'abord attentivement les conseils donnés dans \cite{Melot2007UMONS}.

\newpage

\chapter{Titre du deuxième chapitre}\label{ch:1}
\renewcommand{\leftmark}{CHAPITRE \thechapter.~~Titre du deuxième chapitre}

\section{Section}
Ici on a la première section du chapitre~\ref{ch:1}.

\subsection{Une sous section}
\subsection{Encore une sous section}

\section{Encore une section}

\chapter{Titre du troisième chapitre}
\renewcommand{\leftmark}{CHAPITRE \thechapter.~~Titre du troisième chapitre}

\chapter*{Conclusion}
\addcontentsline{toc}{chapter}{Conclusion}
\renewcommand{\leftmark}{CONCLUSION}

Mettez votre conclusion ici.

% Bibliographie
\bibliographystyle{plain} % 'latex8' est souvent un fichier .bst externe, 'plain' est plus sûr pour tester
\bibliography{biblio}

\newpage
\appendix
\addcontentsline{toc}{chapter}{Annexes}

\chapter{Première annexe}
\renewcommand{\leftmark}{ANNEXE \thechapter.~~Première annexe}
\label{annexe1}

\chapter{Deuxième annexe}
\renewcommand{\leftmark}{ANNEXE \thechapter.~~Deuxième annexe}
\label{annexe2}

\end{document}