\documentclass[12pt,a4paper]{article}

% Langue
\usepackage[french]{babel}

% LuaLaTeX
\usepackage{fontspec}
\setmainfont{Latin Modern Roman}

% Maths
\usepackage{amsmath, amssymb}

% Mise en page
\usepackage{geometry}
\geometry{margin=2.5cm}

% Graphiques
\usepackage{graphicx}
\usepackage{mwe} % <-- pour example-image

% Liens
\usepackage{hyperref}
\usepackage{blindtext}

\hypersetup{
    colorlinks=true,
    linkcolor=black, % ou blue si vous préférez
    urlcolor=blue,
    pdftitle={Votre titre ici}
}

\title{Exemple de document \LaTeX{}}
\author{Alpha Baldé}
\date{\today}

\begin{document}
\maketitle

\tableofcontents
\newpage  
\section{Introduction}
Ceci est un exemple de document \LaTeX{}.

\section{Mathématiques}
\[
\int_{-\infty}^{\infty} e^{-x^2} \, dx = \sqrt{\pi}
\]

\section{Graphiques}
\begin{figure}[h]
    \centering
    \includegraphics[width=0.5\textwidth]{example-image}
    \caption{Un exemple de graphique}
\end{figure}

\Blindtext

\section{Conclusion}
\Blindtext
\section{Annexes}
\Blindtext

\section{liens} 
\href{https://www.overleaf.com/learn/latex/Hyperlinks}{Documentation sur les hyperliens}

\end{document}
